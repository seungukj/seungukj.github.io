%% start of file `template.tex'.
%% Copyright 2006-2012 Xavier Danaux (xdanaux@gmail.com).
%
% This work may be distributed and/or modified under the
% conditions of the LaTeX Project Public License version 1.3c,
% available at http://www.latex-project.org/lppl/.


\documentclass[11pt,a4paper,sans]{moderncv}   % possible options include font size ('10pt', '11pt' and '12pt'), paper size ('a4paper', 'letterpaper', 'a5paper', 'legalpaper', 'executivepaper' and 'landscape') and font family ('sans' and 'roman')

% moderncv themes
\moderncvstyle{casual}                        % style options are 'casual' (default), 'classic', 'oldstyle' and 'banking'
\moderncvcolor{green}                          % color options 'blue' (default), 'orange', 'green', 'red', 'purple', 'grey' and 'black'
%\renewcommand{\familydefault}{\sfdefault}    % to set the default font; use '\sfdefault' for the default sans serif font, '\rmdefault' for the default roman one, or any tex font name
%\nopagenumbers{}                             % uncomment to suppress automatic page numbering for CVs longer than one page

% character encoding
%\usepackage[utf8]{inputenc}                  % if you are not using xelatex ou lualatex, replace by the encoding you are using
\usepackage{CJKutf8}                         % if you need to use CJK to typeset your resume in Chinese, Japanese or Korean
\usepackage{amssymb}

% adjust the page margins
\usepackage[scale=0.75]{geometry}
%\setlength{\hintscolumnwidth}{3cm}           % if you want to change the width of the column with the dates
%\setlength{\makecvtitlenamewidth}{10cm}      % for the 'classic' style, if you want to force the width allocated to your name and avoid line breaks. be careful though, the length is normally calculated to avoid any overlap with your personal info; use this at your own typographical risks...

% personal data
\firstname{Seung uk}
\familyname{Jang}
%\title{장승욱, 張勝旭}                          % optional, remove / comment the line if not wanted
\address{Department of Mathematics, University of Chicago}{5734 S University Ave, Chicago, IL, 60637, USA}    % optional, remove / comment the line if not wanted
%%%%%%%\address{Eckhart 28B, University of Chicago}{1118 E58th Street, Chicago, IL, 60637, USA}    % optional, remove / comment the line if not wanted
%\mobile{+82~(10)~9705~3819}                     % optional, remove / comment the line if not wanted
%\phone{+82~(42)~828~5815}                      % optional, remove / comment the line if not wanted
%\fax{+3~(456)~789~012}                        % optional, remove / comment the line if not wanted
\email{seungukj@uchicago.edu}                          % optional, remove / comment the line if not wanted
%\homepage{blog.naver.com/910217j (facebook: /910217j)}                    % optional, remove / comment the line if not wanted
%\extrainfo{additional information}            % optional, remove / comment the line if not wanted
%\photo[64pt][0.4pt]{Seung_Uk_Jang.jpg}                  % optional, remove / comment the line if not wanted; '64pt' is the height the picture must be resized to, 0.4pt is the thickness of the frame around it (put it to 0pt for no frame) and 'picture' is the name of the picture file
%\quote{알 게 뭐야.}                            % optional, remove / comment the line if not wanted

% to show numerical labels in the bibliography (default is to show no labels); only useful if you make citations in your resume
%\makeatletter
%\renewcommand*{\bibliographyitemlabel}{\@biblabel{\arabic{enumiv}}}
%\makeatother

% bibliography with mutiple entries
%\usepackage{multibib}
%\newcites{book,misc}{{Books},{Others}}
%----------------------------------------------------------------------------------
%            content
%----------------------------------------------------------------------------------
\begin{document}
\begin{CJK*}{UTF8}{gbsn}                     % to typeset your resume in Chinese using CJK
%-----       resume       ---------------------------------------------------------
\makecvtitle

%\vspace*{-5\baselineskip}

%APPLING DEPARTMENT: MATHEMATICS (PH.D)

\section{Contact Details}
\cvitem{Address}{Department of Mathematics, The University of Chicago; 5734 S University Ave, Chicago, IL, 60637, USA}
\cvitem{e-Mail}{seungukj@uchicago.edu}
%\cvitem{Mobile}{+82 (10) 9705-3819}

\section{Education}
\cventry{2014--present}{Ph.D student of Math}{University of Chicago}{Chicago, IL}{\textit{USA}}{Mathematics \\ Advisor: Alex Eskin}
\cventry{2021}{Master of Science}{University of Chicago}{Chicago, IL}{\textit{USA}}{Mathematics}
\cventry{2013--2014}{Master of Science}{KAIST}{Daejeon}{\textit{South Korea}}{Mathematical Sciences}  % arguments 3 to 6 can be left empty
\cventry{2009--2013}{Bachelor of Science}{KAIST}{Daejeon}{\textit{South Korea}}{Mathematical Sciences, with minor on Computer Science, and Honor program \\ \textit{Summa cum Laude}, 4.25/4.3 in GPA \\ \textbf{Valedictorian} for graduate ceremony}
\cventry{2006--2009}{High School}{Korea Science Academy of KAIST}{Busan}{\textit{South Korea}}{}

%\section{Research Interests}
%\cvitem{}{Algebraic geometry on Boolean rings; Analytic number theory}

%\section{Master thesis}
%\cvitem{title}{\emph{Title}}
%\cvitem{supervisors}{Supervisors}
%\cvitem{description}{Short thesis abstract}

\section{Publications}
\cventry{2021}{Kummer Rigidity for Hyperk\"ahler Automorphisms}{Seung uk Jang}{}{}{available as a preprint in \texttt{arXiv:2109.06722}}
\cventry{2017}{Quantum unique ergodicity and the number of nodal domains of eigenfunctions}{Seung uk Jang, Junehyuk Jung}{}{}{Electronically published in J. Amer. Math. Soc., 2017. 06. 02., \\ \texttt{http://dx.doi.org/10.1090/jams/883}}


\section{Academic Talks}
\cventry{2019}{A Mechanical model for Lorenz System}{KIAS}{}{}{introductary material for the Lorenz system and its analysis}
\cventry{2015}{Quantum ergodicity and the number of nodal domains of eigenfunctions}{SNU}{}{}{work done with Junehyuk Jung}
\cventry{2009}{Lattice Edge Number of Figure Eight Knot}{2009 KMS-AMS Joint Meeting}{poster session}{}{work done with Hun Kim, Gyo Taek Jin, Choon Bae Jeon, Sang Hyuk Moon, Sang Hyun Park, Yoo Shin Song}
\cventry{2007}{Generalizing 2D Geometric Properties to 3D With the Aid of DGS}{ATCM 2007}{contributed talks}{}{work done with Dohyun KIM, Hyobin LEE, Youngdae KIM}

%\section{Extracurricular Activities}
%\cventry{Jan 2012 --Dec 2013}{Organizer of Guerrilla Seminar}{KAIST}{}{}{First year graduate students' math seminar series \\ Founding member}
%\cventry{Sep 2010 --Sep 2013}{KAIST Integration of Natural Science, the Natural Science Club}{KAIST}{}{}{Science club of KAIST, ranged on Math, Physics, Chemistry and Biology \\ Founding member, Journal editorial board for Jan 2011--Sep 2012}
%\cventry{Feb 2009 --Dec 2010}{Members of Mathematics, Math Problem Solving Club}{KAIST}{}{}{Club Seminar organizer in Feb 2010--Dec 2010}

\section{Experience}
\subsection{Teaching}
\cventry{Fall 2019 -- Spring 2021}{Lecturer}{University of Chicago}{Chicago}{}{%
Graduate Student Lecturer for various courses, including Freshman Calculus Course (Math 151-153) and \emph{Mathematical Methods for Social Sciences} (Math 195)
}
\subsection{Other Employments}
\cventry{Fall 2016 -- Summer 2019}{Researcher}{NIMS}{Daejeon}{}{
(Division) Center for Applications of Mathematical Principles\newline{}
Employment as an Alternative military service for Korea\newline{}
Working on public understanding of (industrial) mathematics in Korea, including
\begin{itemize}
\item public lectures, generally towards 7th-12th grades students,
\item running and maintaining IMAGINARY exhibitions in Korea, and
\item exploring and developing new items in mathematics that can appeal to general public.
\end{itemize}}
%\subsection{Vocational}
%\cventry{2013--year}{Job title}{Employer}{City}{}{General description no longer than 1--2 lines.\newline{}%
%Detailed achievements:%
%\begin{itemize}%
%\item Achievement 1;
%\item Achievement 2, with sub-achievements:
%  \begin{itemize}%
%  \item Sub-achievement (a);
%  \item Sub-achievement (b), with sub-sub-achievements (don't do this!);
%    \begin{itemize}
%    \item Sub-sub-achievement i;
%    \item Sub-sub-achievement ii;
%    \item Sub-sub-achievement iii;
%    \end{itemize}
%  \item Sub-achievement (c);
%  \end{itemize}
%\item Achievement 3.
%\end{itemize}}
%\cventry{year--year}{Job title}{Employer}{City}{}{Description line 1\newline{}Description line 2}
\subsection{College Fellows}
\cventry{Spring 2016}{Basic Theory of Partial Differential Equations}{Dr. Will Feldman}{U of Chicago}{}{}
\cventry{Winter 2016}{Basic Theory of Ordinary Differential Equations}{Prof. Amie Wilkinson}{U of Chicago}{}{}
\cventry{Fall 2015}{Complex Analysis}{Prof. Amie Wilkinson}{U of Chicago}{}{}
\subsection{Grader}
\cventry{Spring 2014}{Real Analysis}{Prof. Ji Oon Lee}{KAIST}{}{}
\cventry{Fall 2012}{Functional Analysis}{Prof. Ji Oon Lee}{KAIST}{}{}
\subsection{Seminar Organizer}
\cventry{2012--2013}{Guerrilla Seminar}{KAIST}{}{}{First year graduate students' math seminar series \\ Founding member}
\cventry{2010}{Undergraduate Math Colloquium}{KAIST}{}{}{}
%\subsection{Teaching Assistant (non-academic)}
%\cventry{Fall 2011}{Exciting Campus Life}{Office of Freshmen Programs}{KAIST}{}{Recreational activity program for freshmen of KAIST}
%\cventry{Spring 2011}{Happy Campus Life}{Office of Freshmen Programs}{KAIST}{}{Recreational activity program for freshmen of KAIST}

\section{Honors and Awards}
\cventry{2014--present}{Doctorial Study Abroad Program}{}{}{}{Korea Foundation for Advanced Studies}
\cventry{2013--2014}{Kwanjeong Scholarship for Korean Graduate Students}{}{}{}{Scholarship program for graduate students in Korea }%\\ Awarded KRW 11,000,000 per year}
%\cventry{Dec 2012}{Best Senior Thesis Presentation Award}{}{}{}{Department of Mathematical Sciences, KAIST}
\cventry{2011}{Dean's list}{}{}{}{College of Natural Science, KAIST}
%\cventry{Nov 2011}{Gold Prize, University Students Contest of Mathematics in Korea}{}{}{}{Korean Mathematical Society}
%\cventry{Nov 2010}{Silver Prize, University Students Contest of Mathematics in Korea}{}{}{}{Korean Mathematical Society}
\cventry{2009--2012}{Korea Student Aid Foundation, Presidental Scholarship}{}{}{}{Scholarship program for undergraduate students in Korea }%\\ Awarded KRW 7,000,000 per year}

%\section{Languages}
%\cvitemwithcomment{Korean}{(Mother language)}{}
%\cvitemwithcomment{English}{TOEFL iBT 101}{test in 2013/06}
%\cvitemwithcomment{Japanese}{JLPT N1}{test in 2010/08}


%\section{Programming Languages}
%\cvitemwithcomment{Primary}{Java}{Skilled for mathematical codings}
%\cvitemwithcomment{Secondary}{C/C++}{Experience on parallel programming, including CUDA}
%%\cvdoubleitem{category 1}{XXX, YYY, ZZZ}{category 4}{XXX, YYY, ZZZ}
%%\cvdoubleitem{category 2}{XXX, YYY, ZZZ}{category 5}{XXX, YYY, ZZZ}
%%\cvdoubleitem{category 3}{XXX, YYY, ZZZ}{category 6}{XXX, YYY, ZZZ}



%\renewcommand{\listitemsymbol}{-~}            % change the symbol for lists

%\section{Extra 2}
%\cvlistdoubleitem{Item 1}{Item 4}
%\cvlistdoubleitem{Item 2}{Item 5\cite{book1}}
%\cvlistdoubleitem{Item 3}{}

% Publications from a BibTeX file without multibib
%  for numerical labels: \renewcommand{\bibliographyitemlabel}{\@biblabel{\arabic{enumiv}}}
%  to redefine the heading string ("Publications"): \renewcommand{\refname}{Articles}
%------------- former publication section (up to 17. 03. 02)
%\nocite{*}
\newcommand{\MYCOMMENT}[1]{}
%\MYCOMMENT{\cite{dohyunKim1}\cite{JJ15}}
%\bibliographystyle{amsalpha}
%\bibliography{Seung_Uk_Jang}                   % 'publications' is the name of a BibTeX file
%------------- current publication section (from 17. 03. 03)


% Publications from a BibTeX file using the multibib package
%\section{Publications}
%\nocitebook{book1,book2}
%\bibliographystylebook{plain}
%\bibliographybook{publications}              % 'publications' is the name of a BibTeX file
%\nocitemisc{misc1,misc2,misc3}
%\bibliographystylemisc{plain}
%\bibliographymisc{publications}              % 'publications' is the name of a BibTeX file

%\vspace*{\baselineskip}
%\cvitem{Citizenship}{South Korea}
%\section{References}
%\cvitem{Gyo Taek Jin}{Department of Mathematical Sciences, KAIST, Daejeon, South Korea}
%\cvitem{Jinhyun Park}{Department of Mathematical Sciences, KAIST, Daejeon, South Korea}
%\cvitem{Si Jong Kwak}{Department of Mathematical Sciences, KAIST, Daejeon, South Korea}

\MYCOMMENT{%MYCOMMENT
\section{Appendum of Extracurricular Activities}
\cventry{Jan 2012 --Dec 2013}{Organizer of Guerrilla Seminar}{KAIST}{}{}{
Invited talks: \\
(Nov 16, 2013) A Survey on Category Theory \\
(Oct 06, 2013) Logic (Boolean logic), Algebra (Boolean ring), and Topology (Stone spaces) \\
(Sep 14, 2013) Spectrum of a Boolean ring \\
(Aug 29, 2013) Can we get  infinitely many primes by writing two consecutive integers along? \\
(Aug 06, 2013) Riemann's theory on the Prime counter \\
(Jul 31, 2013) Modular forms \\
(Jun 27, 2013) Topology on Fundamental Group and its Galois Connection \\
(Jun 25, 2013) Group cohomology and K(G,1) \\
(Jun 24, 2013) Uniformity on Local rings \\
(Dec 02, 2012) Boolean algebra on 1st order theory and Algebraic representation of G\"odel's incompleteness \\
(Jun 12, 2012) Cauchy's theorem via Sheaf \\
(Jun 11, 2012) Elliptic operator, a list of topics \\
(Jan 20, 2012) Classification theorem of compact 2-dimensional real manifolds \\
(Jan 18, 2012) Riemann surface of Holomorphic functions \\
(Jan 12, 2012) Definition of Simplicial Homology \\
(Jan 12, 2012) Repersentation theory: Unitary presentation and Orthogonality \\
(Jan 09, 2012) Representation theory by Dummit \& Foote Ch. 18-19 \\
(Jan 03, 2012) Definition of p-adic numbers and Hasse principle}
\cventry{Sep 2010 --Sep 2013}{KAIST Integration of Natural Science, the Natural Science Club}{KAIST}{}{}{
Invited talks: \\
(Mar 14, 2012) Calculation: from its Mathematical Foundation to Physical Implementation \\
(Oct 02, 2010) Numerical methods: (IVP) ODE, FDM, and FEM \\ \\
Articles: \\
(Vol. 5: Sep 13, 2013) Riemann's theory on the Prime counter \\
(Vol. 4: Mar 13, 2013) Calculation \\
(Vol. 3: Sep 06, 2012) Hanna Neumann conjecture \\
(Vol. 2: Feb 02, 2012) Representation Theory \\
(Vol. 1: Jan 24, 2011) Numerical Analysis: (IVP) ODE, FDM, FEM \\
(Vol. 1: Jan 24, 2011) Proof of the Multiplication Table}
}%MYCOMMENT
\clearpage

%-----       letter       ---------------------------------------------------------
% recipient data
%\recipient{Company Recruitment team}{Company, Inc.\\123 somestreet\\some city}
%\date{January 01, 1984}
%\opening{Dear Sir or Madam,}
%\closing{Yours faithfully,}
%\enclosure[Attached]{curriculum vit\ae{}}     % use an optional argument to use a string other than "Enclosure", or redefine \enclname
%\makelettertitle
%
%Lorem ipsum dolor sit amet, consectetur adipiscing elit. Duis ullamcorper neque sit amet lectus facilisis sed luctus nisl iaculis. Vivamus at neque arcu, sed tempor quam. Curabitur pharetra tincidunt tincidunt. Morbi volutpat feugiat mauris, quis tempor neque vehicula volutpat. Duis tristique justo vel massa fermentum accumsan. Mauris ante elit, feugiat vestibulum tempor eget, eleifend ac ipsum. Donec scelerisque lobortis ipsum eu vestibulum. Pellentesque vel massa at felis accumsan rhoncus.
%
%Suspendisse commodo, massa eu congue tincidunt, elit mauris pellentesque orci, cursus tempor odio nisl euismod augue. Aliquam adipiscing nibh ut odio sodales et pulvinar tortor laoreet. Mauris a accumsan ligula. Class aptent taciti sociosqu ad litora torquent per conubia nostra, per inceptos himenaeos. Suspendisse vulputate sem vehicula ipsum varius nec tempus dui dapibus. Phasellus et est urna, ut auctor erat. Sed tincidunt odio id odio aliquam mattis. Donec sapien nulla, feugiat eget adipiscing sit amet, lacinia ut dolor. Phasellus tincidunt, leo a fringilla consectetur, felis diam aliquam urna, vitae aliquet lectus orci nec velit. Vivamus dapibus varius blandit.
%
%Duis sit amet magna ante, at sodales diam. Aenean consectetur porta risus et sagittis. Ut interdum, enim varius pellentesque tincidunt, magna libero sodales tortor, ut fermentum nunc metus a ante. Vivamus odio leo, tincidunt eu luctus ut, sollicitudin sit amet metus. Nunc sed orci lectus. Ut sodales magna sed velit volutpat sit amet pulvinar diam venenatis.
%
%Albert Einstein discovered that $e=mc^2$ in 1905.
%
%\[ e=\lim_{n \to \infty} \left(1+\frac{1}{n}\right)^n \]
%
%\makeletterclosing

\clearpage\end{CJK*}                         % if you are typesetting your resume in Chinese using CJK; the \clearpage is required for fancyhdr to work correctly with CJK, though it kills the page numbering by making \lastpage undefined
\end{document}


%% end of file `template.tex'.
